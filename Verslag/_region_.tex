\message{ !name(chua_circuit.tex)}\documentclass[twocolumn,10pt]{article}

\usepackage{amsmath,amssymb,amsthm}
\usepackage{graphicx}

\title{On the Chua Circuit}
\author{Rosa Kappert \and Eric Spreen \and Johanna Stegink}

\begin{document}

\message{ !name(equilibria.tex) !offset(-11) }
\section{Investigating the equilibria}

First, we will locate the equilibria of this system.  Note that we
will take the parameters $a, b, \rho,\sigma > 0$.  This is because the
first two correspond to physical parameters that are positive and the
other two are positive by assumption.

Equating the right part of equation \ref{eq:system} with zero, we get
from the $z'$ part, noting that $b \neq 0$, that $y = 0$.
Furthermore, the second equation tells us then that $z = -x$.  And
lastly, the first equation tells us to look for the roots of
$\phi(x)$.  We will factor this into:
\begin{equation}
  \label{eq:phi-factor}
  \phi(x) = x ( \rho x^2 - \sigma ) = 0,
\end{equation}
from which we deduce that $x = 0$ or $x = \pm
\sqrt{\frac{\sigma}{\rho}}$.

This leaves us with three equilibrium points for the system:
\begin{gather}
  X_0 = (x,y,z)_0 = (0,0,0)\\
  X_\pm =(x,y,z)_\pm = (\pm\sqrt{\frac{\sigma}{\rho}}, 0,
  \mp\sqrt{\frac{\sigma}{\rho}}),
\end{gather}
one equilibrium in the origin and one in the $xz$-plane, symmetric in
the origin.  Note that the location of the equilibria $X_\pm$ is
completely determined by the parameters in $\phi$, and that these fail
to exist if $\rho$ and $sigma$ do not have the same sign.

\subsection{Classification of the origin}

We will now linearize the system about the equilibria and classify
these points on this.  For $X_0$ (the origin), we have the following
linearization:
\begin{gather}
  \label{eq:linear-origin}
  X' = A_0 X, \\
  A_0 :=
  \begin{pmatrix}
    -a \phi'(0) & a & 0\\
    1 & -1 & 1\\
    0 & -b & 0
  \end{pmatrix}
  =
  \begin{pmatrix}
    a \sigma & a & 0\\
    1 & -1 & 1\\
    0 & -b & 0
  \end{pmatrix}.
\end{gather}
Now, to study the eigenvalues of $A_0$, we note that $tr(A_0) = a \sigma
- 1$ and $det(A_0) = a b \sigma > 0$.  For any real $n \times n$ matrix,
we know that the trace is equal to the sum and the
determinant to the product of the eigenvalues, counting
multiplicities.  For the eigenvalues of $A_0$ we have three cases:
\begin{enumerate}
\item All eigenvalues are zero
\item All eigenvalues are real and distinct
\item One eigenvalue is real and two are complex and non-real.
\end{enumerate}

Bare with us, because this will get confusing.

\paragraph{Zero eigenvalues}
This case is fundamentally impossible, since in this case, we would
have $0 = det(A_0) = ab\sigma > 0$, which is a contradiction.
Therefore, all eigenvalues are non-zero.

\paragraph{Real and distinct eigenvalues}
Let $\lambda_i \in \mathbb{R} (i=1,2,3)$ be the eigenvalues of $A_0$.
Then we have $0 < ab\sigma = det(A_0) = \lambda_1\lambda_2\lambda_3$.
If without generality we assume that $\lambda_1 < \lambda_2 <
\lambda_3$, we can again distinguish two cases.  If $\lambda_1 < 0$,
we must have $\lambda_2 < 0$ and $\lambda_3 > 0$.  This corresponds to
a saddle with a stable plane.

If $\lambda_1 > 0$, we automatically have $\lambda_i > 0 (i=1,2,3)$.
This corresponds to a source.

Now we investigate the influence of the parameters $a,b,\sigma$ on the
eigenvalues.  If $0 < a\sigma \leq 1$, we see that:
\begin{gather*}
  0 \geq a\sigma - 1 = tr(A_0) = \sum_{i=1}^3 \lambda_i.
\end{gather*}
Therefore:
\begin{gather*}
  \lambda_3 \leq -(\lambda_1 + \lambda_2)
\end{gather*}
So, if $\lambda_1 > 0$, we get $0 < \lambda_3 \leq 0$, which can't be
true.  Therefore we must have in this case (i.e. if there are only
real eigenvalues and $0 < a\sigma \leq 1$) that $\lambda_1,\lambda_2 <
0$ and $\lambda_3 \leq |\lambda_1 + \lambda_2|$.  So, then the origin
is a saddle with a stable plane.

Now to study the subcase of $a\sigma > 1$, we observe that if
$\lambda_1,\lambda_2 < 0$, we get $\lambda_3 > - (\lambda_1 +
\lambda_2) = |\lambda_1 + \lambda_2| > 0$.  And if
$\lambda_1,\lambda_2>0$, we get $\lambda_3 > -(\lambda_1 +
\lambda_2)$, which is perfectly possible, since we know that
$\lambda_3 > 0$.  Therefore, on these criteria, both cases (source and
saddle) can occur.

\paragraph{One real, two non-real eigenvalues}
Again let $\lambda_1\in\mathbb{R}, \lambda_2,\lambda_3\in\mathbb{C}$
be the eigenvalues of $A_0$.  Then we immediately know that $\lambda_3
= \bar{\lambda_2}$, the complex conjugate of $\lambda_2$.  From the
trace and determinant we deduce:
\begin{gather*}
  0 < ab\sigma = \lambda_1 |\lambda_2|^2 \implies \lambda_1 > 0\\
  a\sigma - 1 = \lambda_1 + \lambda_2 + \bar{\lambda_2} 
  = \lambda_1 + 2 Re(\lambda_2).
\end{gather*}
So we see that the real eigenvalue must be strictly positive (since
$|\lambda_2|^2 > 0$. Furthermore, if $0< a\sigma \leq 1$, the real part
of $\lambda_2$ and $\lambda_3$ is strictly negative.

When $a \sigma > 1$, the only restriction we can impose is that the
real part of $\lambda_2$ and $\lambda_3$ is strictly greater than
$-\frac{\lambda_1}{2}$.

\paragraph{Type of equilibrium}

From this we can deduce that as long as $a \leq \frac{1}{\sigma}$, the
origin will be a saddle point, with a stable plane.  When $a$ becomes
greater than $\frac{1}{\sigma}$, there is a possible bifurcation,
changing the origin into a source.

\subsection{Classification of $X_\pm$}



%%% Local Variables: 
%%% mode: latex
%%% TeX-master: "chua_circuit"
%%% End: 

\message{ !name(chua_circuit.tex) !offset(-132) }

\end{document}
