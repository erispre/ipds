\section{Bifurcations}

We already showed that the origin is strictly a saddle point. To see
what types of bifurcations might occur in the system when a is varied,
we ran a simulation with an initial point close to one of the
equilibrium points away from the origin and had a vary from 4 to 14.

The equilibrium point we started near acts as a spiral sink for low
values of; $4 \leq a \leq 6$. When $6 \leq a \leq 6.64$, the solution
orbit approaches a circular limit set around the equilibrium
point. This limit set seems to disappear for $6.64 \leq a \leq 8.3$,
but is probably just too small to see, for these values of a, the
orbit approaches the equilibrium point, circles it and moves away to a
larger limit set surrounding it, which implies that the equilibrium
point is no longer a spiral sink, but rather a saddle point of some
kind. The outer orbit limit is drawn to the origin, eventually
splitting it into two orbit limits, one of which continues to move
toward the origin as a increases. When $8.68 \leq a \leq 9.37$ the
outer orbit limit is close enough to the origin to be affected by it
and for $a \geq 8.8$ the orbit is pulled through the origin and around
the equilibrium point on the other side.

The circular limit set around the first equilibrium point is more
clearly present around the second equilibrium point, and fluctuates in
size as a increases from 9.37 to 10.7, steadily increasing its size
from $a=10.1$ onwards.

When a increases beyond 10.7, the orbit instantly moves away from its
previous orbit and approaches a new limit set away from all 3
equilibrium points.

For an illustration of these changes, see figure \ref{fig:numapprox},
which shows numerous numerical approximations for $b = 14, \rho =
\frac{1}{16}$ and $\sigma = \frac{1}{6}$ and various values of $a$.
The calculations all started at the same initial point, namely $(1.63,
0, -1.63)$, near one of the equilibria.

%%% Local Variables: 
%%% mode: latex
%%% TeX-master: "chua_circuit"
%%% End: 
