\section{Investigating the equilibria}

First, we will locate the equilibria of this system.  Note that we
will take the parameters $a, b, \rho,\sigma > 0$.  This is because the
first two correspond to physical parameters that are positive and the
other two are positive by assumption.

Equating the right part of equation \ref{eq:system} with zero, we get
from the $z'$ part, noting that $b \neq 0$, that $y = 0$.
Furthermore, the second equation tells us then that $z = -x$.  And
lastly, the first equation tells us to look for the roots of
$\phi(x)$.  We will factor this into:
\begin{equation}
  \label{eq:phi-factor}
  \phi(x) = x ( \rho x^2 - \sigma ) = 0,
\end{equation}
from which we deduce that $x = 0$ or $x = \pm
\sqrt{\frac{\sigma}{\rho}}$.

This leaves us with three equilibrium points for the system:
\begin{gather}
  X_0 = (x,y,z)_0 = (0,0,0)\\
  X_\pm =(x,y,z)_\pm = (\pm\sqrt{\frac{\sigma}{\rho}}, 0,
  \mp\sqrt{\frac{\sigma}{\rho}}),
\end{gather}
one equilibrium in the origin and two in the $xz$-plane, symmetric in
the origin.  Note that the location of the equilibria $X_\pm$ is
completely determined by the parameters in $\phi$, and that these fail
to exist if $\rho$ and $sigma$ do not have the same sign.

\subsection{Classification of the origin}

We will now linearize the system about the equilibria and classify
these points based on this.  For $X_0$ (the origin), we have the
following linearization:
\begin{gather}
  \label{eq:linear-origin}
  X' = A_0 X, \\
  A_0 :=
  \begin{pmatrix}
    -a \phi'(0) & a & 0\\
    1 & -1 & 1\\
    0 & -b & 0
  \end{pmatrix}
  =
  \begin{pmatrix}
    a \sigma & a & 0\\
    1 & -1 & 1\\
    0 & -b & 0
  \end{pmatrix}.
\end{gather}
Now, to study the eigenvalues of $A_0$, we note that $tr(A_0) = a \sigma
- 1$ and $det(A_0) = a b \sigma > 0$.  For any real $n \times n$ matrix,
we know that the trace is equal to the sum and the
determinant to the product of the eigenvalues, counting
multiplicities.  For the eigenvalues of $A_0$ we have three cases:
\begin{enumerate}
\item All eigenvalues are zero
\item All eigenvalues are real and non-zero
\item One eigenvalue is real and two are complex and non-real, and all
  are non-zero.
\end{enumerate}

We can distinguish these cases by studying the characteristic
polynomial, which is given by:
\begin{equation}
  \label{eq:origin-cp}
  p_{A_0}(\lambda) = -\lambda^3+(a\sigma-1)\lambda
  +(a\sigma + a -b) \lambda + a b \sigma.
\end{equation}

\paragraph{Zero eigenvalues}
This case is fundamentally impossible, since in this case, we would
have $0 = det(A_0) = ab\sigma > 0$, which is a contradiction.
Therefore, all eigenvalues are non-zero.

\paragraph{Real and non-zero eigenvalues}
We will study this case with the help of the computer algebra program
Wolfram Mathematica.  Using the script in appendix
\ref{app:mathematica}, we can compare the coefficients of the
characteristic polynomial with a factorization that would arize when
all roots are real and non-zero.  We hereby note that either all
eigenvalues must be positive, or two must be negative and one
positive, since the determinant is positive.  The automated algebraic
analysis shows that the former case is impossible, so in this case we
must have two negative eigenvalues and one positive.

We can prove this more rigorously for the subcase $0 < a\sigma \leq
1$. Let $\lambda_i \in \mathbb{R} (i=1,2,3)$ be the eigenvalues of
$A_0$.  Then we have $ab\sigma = det(A_0) =
\lambda_1\lambda_2\lambda_3 > 0$.  If without generality we assume
that $\lambda_1 \leq \lambda_2 \leq \lambda_3$, we can again
distinguish two cases.  If $\lambda_1 < 0$, we must have $\lambda_2 <
0$ and $\lambda_3 > 0$.  This corresponds to a saddle with a stable
plane, spanned by the (generalized) eigenvectors corresponding to
$\lambda_1$ and $\lambda_2$.

If $\lambda_1 > 0$, we automatically have $\lambda_i > 0 (i=1,2,3)$.
This corresponds to a source.

Now we investigate the influence of the parameters $a,b,\sigma$ on the
eigenvalues.  If $0 < a\sigma \leq 1$, we see that:
\begin{gather*}
  0 \geq a\sigma - 1 = tr(A_0) = \sum_{i=1}^3 \lambda_i.
\end{gather*}
Therefore:
\begin{gather*}
  \lambda_3 \leq -(\lambda_1 + \lambda_2)
\end{gather*}
So, if $\lambda_1 > 0$, we get $0 < \lambda_3 \leq 0$, which can't be
true.  Therefore we must have in this case (i.e. if there are only
real eigenvalues and $0 < a\sigma \leq 1$) that $\lambda_1,\lambda_2 <
0$ and $\lambda_3 \leq |\lambda_1 + \lambda_2|$.  So, then the origin
is a saddle with a stable plane.

Now to study the subcase of $a\sigma > 1$, we observe that if
$\lambda_1,\lambda_2 < 0$, we get $\lambda_3 > - (\lambda_1 +
\lambda_2) = |\lambda_1 + \lambda_2| > 0$.  And if
$\lambda_1,\lambda_2>0$, we get $\lambda_3 > -(\lambda_1 +
\lambda_2)$, which is perfectly possible, since we know that
$\lambda_3 > 0$.  Therefore, on these criteria, both cases (source and
saddle) can occur.

Note that if we include the third requirement $-(\lambda_1\lambda_2 +
\lambda_2 \lambda_3 + \lambda_1\lambda_3 = a-b+a\sigma$, which arises
from equating $p_{A_0}(\lambda)$ with its factorization, we can proof
it also for this subcase.  However, we feel that such a proof is mere
very precise bookkeeping, which can best be done by a computer algebra
system.

\paragraph{One real, two non-real eigenvalues}
Now, let $\lambda_1\in\mathbb{R}, \lambda_2,\lambda_3\in\mathbb{C}$
be the eigenvalues of $A_0$.  Then we immediately know that $\lambda_3
= \bar{\lambda_2}$, the complex conjugate of $\lambda_2$.  From the
trace and determinant we deduce:
\begin{gather*}
  0 < ab\sigma = \lambda_1 |\lambda_2|^2 \implies \lambda_1 > 0\\
  a\sigma - 1 = \lambda_1 + \lambda_2 + \bar{\lambda_2} 
  = \lambda_1 + 2 Re(\lambda_2).
\end{gather*}
So we see that the real eigenvalue must be strictly positive (since
$|\lambda_2|^2 > 0$. Furthermore, if $0< a\sigma \leq 1$, the real part
of $\lambda_2$ and $\lambda_3$ is strictly negative.

When $a\sigma > 1$, we can again factor the characteristic polynomial:
\begin{gather*}
  \label{eq:origin-cp-factor}
  p_{A_0}(\lambda) = k(\lambda - (\mu + \nu i))(\lambda - (\mu - \nu
  i))(\lambda - z) \\
  \quad (k,\mu,\nu,z \in \mathbb{R}).
\end{gather*}
Again we can let Mathematica do the algebraic work, from which we get
$Re(\lambda_1) = Re(\lambda_2) < 0$ for all (positive) values of the
parameters.

\paragraph{Type of equilibrium}

So, we can deduce that in every case we will always have two
eigenvalues with negative real part and one real positive eigenvalue.
The question now arises whether the eigenvalues are real, or two are
non-real.  We can determine this by looking at the cubic discriminant
of the characteristic polynomial.  The algebraic expression for this
is too large to include in this report, but we can tell you that it
is a continuous function of the parameters.  We can therefore fix some
parameter values and then check the behaviour of the eigenvalues near
these points.

With this in mind, we will now fix $b = 14, \rho = \frac{1}{16}$ and
$\sigma = \frac{1}{6}$.  We can now determine the roots of the
discriminant with respect to $a$.  A numerical calculation shows that
two roots are real and negative and the other two are non-real.  Now,
we can calculate the discriminant for one positive value of $a$,
e.g. $a = 1$.  This gives a negative value of the discriminant, so we
can conclude that for these parameters and all positive values of $a$,
we get a negative discriminant, which corresponds to two non-real and
one real eigenvalue.

\subsection{Classification of $X_\pm$}

We follow the same road as with the origin.  The linearized system is:
\begin{gather}
  \label{eq:linear_pm}
  Y' = A_\pm Y\\
  A_\pm :=
  \begin{pmatrix}
    -2 a \sigma & a & 0\\
    1 & -1 & 1\\
    0 & -b & 0
  \end{pmatrix}.
\end{gather}
We now have $det(A_\pm) = -2a \sigma b < 0$ and $tr(A_\pm) = -(2a\sigma +
1) < 0$ and the same three possibilities as above.  Again, we
immediately note that an eigenvalue zero is not possible.

\paragraph{Real eigenvalues}

We first note that because $-2ab\sigma < 0$, we must have either one
or three negative eigenvalues.  A numerical search shows that there
are instances of the parameters available for each of these cases.  We
therefore conclude that both cases are possible: a one-dimensional
stable manifold with a two-dimensional unstable manifold (saddle) and
a three-dimensional stable manifold (a sink).

\paragraph{Two non-real, one real eigenvalues}

This case turns out to be troubling.  It is possible to find values of
the parameters such that the real part of the non-real eigenvalues is
zero.  This indicates that for these parameter values this method of
classification is undetermined.

We can however conclude that the real eigenvalue is always negative
(because of the negative determinant).  But apart from that, the real
part of the non-real eigenvalues can be negative, positive and zero.

\paragraph{Type of eigenvalues}

We can of course again use the discriminant of the characteristic
polynomial to determine whether the eigenvalues will be real or
non-real.  Again we fix the parameters as in the case of the origin
and using the same methods we find a negative discriminant for all
positive values of $a$, in a neighbourhood of $b = 14, \sigma =
\frac{1}{6}$ and $\rho = \frac{1}{16}$.

Therefore, in this neighbourhood in the parameter space, we have one
real and two non-real eigenvalues, which gives the troubling behaviour
of above.  I.e. for some parameter values, there may not exist a
conjugation between the original and the linearized system.  One
example is $a = \frac{3}{2}(-1 + \sqrt{29})$.  It is however to be
noted that the eigenvalues depend continuously on the coefficients in
the characteristic polynomial, which depend continously on the
parameters.  Therefore all bifurcations with respect to the real part
of the non-real eigenvalues of the linearized system occur at the
points in the parameter space for which the real part is equal to zero.


%%% Local Variables: 
%%% mode: latex
%%% TeX-master: "chua_circuit"
%%% End: 
